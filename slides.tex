\documentclass{beamer}

\usetheme{default}
\usepackage[utf8]{inputenc}
\usepackage{amsmath}
\usepackage{graphicx}

\title{The German Statutory Pension Insurance (Gesetzliche Rentenversicherung)}
\subtitle{Mechanism Design, Demographic "Greying," and the Rentenpaket II Paradigm Shift}
\author{Richard Schulz \and Jakob Frerichs}
\institute{Supervised by Dr. Cheng Wan, ETH Zurich \\ Population Ageing and Pension Economics (PAPE)}
\date{December 2, 2025}

\begin{document}

\begin{frame}
  \titlepage
\end{frame}

\begin{frame}{Agenda}
  \begin{enumerate}
    \item \textbf{Introduction:} The Economic Trilemma \& Theoretical Foundation
    \item \textbf{System Design:} Institutional Structure, Points System, and Pension Formula
    \item \textbf{Supplementary Pillars:} Occupational and Private Pensions
    \item \textbf{Demographic Challenges:} Aging Population and Labor Market Responses
    \item \textbf{Current Reforms:} Rentenpaket II and the \textit{Generationenkapital}
    \item \textbf{Outlook:} Remaining Problems and Conclusion
  \end{enumerate}
\end{frame}

\begin{frame}{The Economic Trilemma \& Mackenroth's Theorem}
  \textbf{The Fundamental Constraint:} The "Pension Trilemma."
  \begin{itemize}
    \item \textbf{Contribution Rate} (\textit{Beitragssatz}): How much workers pay. Politically capped (approx. 20-22\%)
    \item \textbf{Replacement Rate} (\textit{Rentenniveau}): How much pensioners get. Politically protected (floor at 48\%)
    \item \textbf{Retirement Age}: How long you work. Also politically sensitive
  \end{itemize}
  \textit{You can only fix two; the third must adjust. Germany is trying to fix all three.}
  
  \vspace{1em}
  \textbf{Theoretical Basis:} \textit{Mackenroth’s Theorem (1952)}:
  \begin{quote}
    "All social expenditure must always be paid out of the current national product."
    \\ \small (Goods for retirees are made by current workers, regardless of funding type)
  \end{quote}
\end{frame}

\begin{frame}{Institutional Design: A System Reliant on One Pillar}
  \textbf{Structure:}
  \begin{itemize}
    \item \textbf{Pillar 1 (GRV):} Mandatory Pay-As-You-Go (PAYG)
    \begin{itemize}
        \item Covers $\sim$90\% of the workforce, providing $\sim$75\% of old-age income, but notably EXCLUDES civil servants (\textit{Beamte}), who have a separate, state-financed pension system
        \item \textbf{Result:} A "mono-pillar" system, highly exposed to demographic shifts
    \end{itemize}
    \item \textbf{Pillar 2 (bAV) \& 3 (Private):} Voluntary and have seen limited uptake
  \end{itemize}
  


  \textbf{Financing:}
  \begin{itemize}
    \item Contributions (Employer/Employee split 50/50)
    \item \textbf{Federal Subsidy (\textit{Bundeszuschuss}):} $\sim$20\% of the budget comes from general taxes, not contributions
  \end{itemize}
\end{frame}

\begin{frame}{Pillar 1: The "Points System" (\textit{Entgeltpunkte})}
  \textbf{The Logic:} Your pension reflects your lifetime relative income, it's not a "final salary" system.
  \begin{itemize}
    \item \textbf{Accumulation Mechanism:}
    \[
      EP_t = \frac{\text{Individual Gross Income}_t}{\text{Average Gross Income}_t}
    \]
    \small{\textbf{Example (2024):} The provisional average income is $\sim$€45,400. Earning this amount gets you 1.0 point. Earning €68,100 gets you 1.5 points}
    \item \textbf{Incentive Structure (The Equivalence Principle):}
    \begin{itemize}
        \item Strict linearity ($Contribution \propto Benefit$)
        \item \textbf{Economic Goal:} Makes contributions feel like "deferred wages," not a tax, to minimize labor supply distortions
    \end{itemize}
  \end{itemize}
\end{frame}

\begin{frame}{The Pension Formula: Turning Points into Euros}
    \textbf{The Equation:}
    \[
        \text{Pension}_{\text{Monthly}} = \sum EP \times ZF \times RAF \times AR
    \]
    \textbf{Variables:}
    \begin{itemize}
        \item $\sum EP$: Your total lifetime earnings points
        \item $ZF$ (Access Factor): An early retirement penalty. 0.3\% permanent cut for every month you retire before the statutory age
        \item $RAF$: Pension Type (1.0 for standard old age)
        \item $AR$ (Current Pension Value): The "exchange rate" for one pension point. As of July 2024, its value is \textbf{€39.32}
    \end{itemize}
\end{frame}

\begin{frame}{Pillar 2: Occupational Pensions (\textit{Betriebliche Altersversorgung - bAV})}
    \textbf{Scope:}
    \begin{itemize}
        \item Company-sponsored pension schemes, typically covering only employees of larger firms
        \item Voluntary system with tax incentives for both employers and employees
    \end{itemize}
    
    \vspace{0.5em}
    \textbf{Relevance:} Despite tax incentives, coverage remains limited:
    \begin{itemize}
        \item Only $\sim$50\% of employees have access to occupational pensions
        \item Coverage is concentrated in large companies and public sector
        \item Small and medium enterprises (SMEs) rarely offer bAV
        \item \textbf{Result:} Provides only $\sim$5-10\% of total old-age income, far below the intended complement to Pillar 1
    \end{itemize}
\end{frame}

\begin{frame}{Pillar 3: Private Pensions (\textit{Riesterrente})}
    \textbf{The Concept:} State-subsidized private pension savings (introduced 2001)
    
    \vspace{0.5em}
    \textbf{Why Low Uptake?}
    \begin{itemize}
        \item \textbf{Complexity:} Multiple product types, confusing eligibility rules, and bureaucratic application processes
        \item \textbf{Low Returns:} High fees and conservative investment strategies erode returns, making it unattractive compared to alternatives
        \item \textbf{Means-Testing Penalty:} For low-income savers, Riester benefits are offset by reductions in \textit{Grundsicherung}, creating a 100\% effective marginal tax rate
        \item \textbf{Trust Deficit:} Public skepticism about private financial products after financial crises
    \end{itemize}
    
    \vspace{0.5em}
    \textbf{Reality:} Only $\sim$16 million contracts (out of 45 million eligible)
\end{frame}

\begin{frame}{The Demographic Time Bomb is Ticking}
  \begin{itemize}
    \item This rule-based static system now faces a massive exogenous shock: demography
    \item By 2035, the last of Germany's "Baby Boomer" generation will have retired
    \item The Old-Age Dependency Ratio (OADR) is projected to soar from 35 to over 50
    \begin{itemize}
        \item \textbf{Today:} $\sim$3 workers support 1 pensioner
        \item \textbf{By 2050:} $\sim$2 workers will have to support 1 pensioner
    \end{itemize}
    \item \textbf{The Fiscal Squeeze:} Without reform, the contribution rate is forecast to rise from 18.6\% to over 24\% by 2040
  \end{itemize}
  \vspace{1em}
  \centering
  \textbf{\large The "Generational Contract" is under unprecedented stress.}
\end{frame}

\begin{frame}{Labor Supply \& The Politics of Retirement Age}
  \begin{itemize}
    \item \textbf{Policy Response: "Rente mit 67" (2007 Reform)}
    \begin{itemize}
        \item Gradually increases the Statutory Retirement Age (SRA) to 67 by 2031 to keep people working longer
    \end{itemize}
    \item \textbf{The Policy Anomaly: "Rente mit 63" (2014)}
    \begin{itemize}
        \item Allowed long-term contributors to retire early without penalty
        \item \textbf{Economic Critique:} A huge "deadweight loss" by subsidizing the exit of highly productive, skilled labor during a growing labor shortage
    \end{itemize}
    \item \textbf{Result:} The effective retirement age (64.7) remains below the statutory age
  \end{itemize}
\end{frame}

\begin{frame}{The Sustainability Factor: Shifting Risk by Dampening Growth}
  \textbf{The Goal:} Automatically adjust for demography
  \begin{itemize}
    \item Given the demographic pressure, the formula has a built-in endogenous adjustment mechanism
    \item \textbf{The Dampener: The Sustainability Factor (enacted 2004)}
    \begin{itemize}
      \item If the dependency ratio (pensioners/workers) worsens, pension increases are "dampened" and do not fully follow wage growth
      \item \textbf{Crucial Distinction:} This factor does \textbf{not} cut nominal pensions. It reduces the annual \textit{rate of increase} of the pension value (AR), making it lag behind national wage growth
      \[
        AR_t \approx AR_{t-1} \times \text{WageGrowth}_t \times (1 - \alpha \cdot \Delta R_t)
      \]
      {\small where $R_t$ is the pensioner-to-contributor ratio}
    \end{itemize}
  \end{itemize}
\end{frame}

\begin{frame}{The "Standard Pensioner" \& Falling Replacement Rates}
  \begin{itemize}
    \item These pressures and adjustments have direct consequences for the key political trade-offs
    \item \textbf{Standard Pensioner (\textit{"Eckrentner"}):} A theoretical person with 45 years of average contributions
    \begin{itemize}
        \item \textbf{Problem:} This ignores fractured careers and part-time work, thus painting a deceptively rosy picture
    \end{itemize}
    \item \textbf{Replacement Rate Trends (The Worry):}
    \begin{itemize}
        \item The net replacement rate is currently held at $\sim$48\%
        \item \textit{Forecast:} The Sustainability Factor would have caused it to drop to 45\% by 2040
    \end{itemize}
    \item \textbf{Contribution Rate Trends (The Other Worry):}
    \begin{itemize}
        \item Currently $\sim$18.6\%
        \item \textit{Forecast:} Set to break the politically sensitive 20\% barrier by 2028
    \end{itemize}
  \end{itemize}
\end{frame}

\begin{frame}{The Social Safety Net}
  \textbf{The Reality for Low-Income Pensioners:}
  \begin{itemize}
    \item \textbf{Problem:} A low statutory pension level can lead to old-age poverty. The average monthly pension for those with 35+ years of contributions ranges from $\sim$€1,460 (women) to $\sim$€1,890 (men)
    \item \textbf{The Floor: Basic Income Support}
    \begin{itemize}
        \item A means-tested social welfare benefit that acts as a floor if the state pension is insufficient to live on
        \item At the end of 2023, \textbf{$\sim$690,000 pensioners} relied on this benefit
    \end{itemize}
    \item \textbf{The Economic Incentive Problem:}
    \begin{itemize}
        \item For low-income workers, private savings (Pillar 3) can be offset by reductions in means-tested benefits, creating a 100\% marginal tax rate on those savings and discouraging private provision
    \end{itemize}
  \end{itemize}
\end{frame}

\begin{frame}{Rentenpaket II: Politics Overrules Economics}
  \begin{itemize}
    \item \textbf{The "Double Stop Line" (\textit{Doppelte Haltelinie}):}
    \begin{itemize}
        \item A political promise: Keep the replacement rate $\ge 48\%$ AND the contribution rate $\le 20\%$
    \end{itemize}
    \item \textbf{The New Deal (Rentenpaket II):}
    \begin{itemize}
        \item This package is the subject of intense current political and economic debate
        \item Permanently guarantees the 48\% level until 2031
        \item \textbf{The Cost:} This deactivates the Sustainability Factor, removing the automatic brake on spending
        \item \textbf{The Consequence:} The fiscal burden is shifted entirely to the federal budget via massive tax subsidies (\textit{Bundeszuschuss})
    \end{itemize}
  \end{itemize}
  \begin{figure}
  \centering
  \includegraphics[width=\textwidth,height=0.4\textheight,keepaspectratio]{political_proposals_chart.png}
  \caption{Proposals of German political parties for the statutory pension system (2021). Source: eniref.org}
  \end{figure}
\end{frame}

\begin{frame}{Paradigm Shift? The \textit{Generationenkapital}}
    \textbf{Concept:} Create a Sovereign Wealth Fund to help subsidize the pension system
    \begin{itemize}
        \item \textbf{Mechanism: "Debt-Financed Arbitrage"}
        \begin{itemize}
            \item The state borrows money at low interest rates (cost of government bonds)
            \item It invests this money in a globally diversified portfolio of stocks (aiming for higher equity returns)
            \item The goal is to profit from the spread: $r_{\text{equity}} > r_{\text{bond}}$
        \end{itemize}
        \item \textbf{Target:} €200bn fund by the mid-2030s
        \item \textbf{Advanced Risk Analysis:}
        \begin{itemize}
            \item \textbf{Too Small:} Experts argue it needs more than €1 Trillion to have a meaningful impact on contribution rates
            \item \textbf{Governance Risks:} Can investment decisions remain free from political interference?
            \item \textbf{Systematic Risk Exposure:} A market crash during a recession creates a pro-cyclical fiscal liability, as the fund and tax revenues fall simultaneously
            \item \textbf{Governance \& Time Inconsistency:} Can a government resist the political temptation to alter investment strategy for short-term goals, compromising long-term returns?
        \end{itemize}
    \end{itemize}
\end{frame}

\begin{frame}{Problems remain}
    \begin{itemize}
        \item \textbf{Fiscal Crowding Out:}
        \begin{itemize}
            \item Pension subsidies already consume $\sim$20\% of the entire German federal budget
            \item This severely limits fiscal space for infrastructure, defense, digitalization, and education
        \end{itemize}
        \item \textbf{The "Boomer Voter" Effect:}
        \begin{itemize}
            \item The large voting bloc of the elderly creates strong political resistance to actuarially necessary cuts (Median Voter Theorem)
        \end{itemize}
        \item \textbf{The Reform Challenge:}
        \begin{itemize}
            \item The German government now wants to create a \textit{Rentenkommission} to find a compromise in the pension problem
            \item Supposedly no constraints on potential solutions
            \item \textbf{The Reality:} It is hard to reform pension systems in demographically changing countries due to political resistance, path dependency, and the long-term nature of pension commitments
        \end{itemize}
    \end{itemize}
\end{frame}

\begin{frame}{Conclusion}
  \textbf{Summary:}
  \begin{itemize}
    \item Germany's PAYG system is efficient in its design but fundamentally vulnerable to its own demographic decline
    \item Recent reforms (\textit{Rentenpaket II}) have prioritized short-term benefit security for current pensioners over long-term fiscal sustainability, effectively passing the bill to future generations
  \end{itemize}
  \textbf{The Outlook:}
  \begin{itemize}
    \item The \textit{Generationenkapital} is a historic step towards capital funding, but it is too small to solve the structural problem
    \item The "Contract between Generations" is being rewritten, with the young bearing the demographic and fiscal risk
  \end{itemize}
  \vspace{1em}
  \centering
  \textbf{\large The final question remains: Who is paying for the pension system?}
\end{frame}

\begin{frame}{References}
  \begin{itemize}
    \item German Council of Economic Experts (Sachverständigenrat)
    \item Deutsche Rentenversicherung (DRV) Reports 2023/2024
    \item OECD Pensions at a Glance 2023
    \item Mackenroth, G. (1952). \textit{Die Reform der Sozialpolitik}
    \item ENIREF European Network for Research on Economic Policy (eniref.org)
    \item Pensionfriend.de
    \item \textit{Note: Comparisons to Swiss and Japanese systems drawn from course materials}
  \end{itemize}
\end{frame}

\end{document}
