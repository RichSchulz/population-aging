\documentclass{beamer}

\usetheme{default}
\usepackage[utf8]{inputenc}
\usepackage{amsmath}

\title{The German Statutory Pension Insurance (Gesetzliche Rentenversicherung - GRV)}
\subtitle{Mechanism Design, Demographic "Greying," and the Rentenpaket II Paradigm Shift}
\author{Richard Schulz \and Jakob Frerichs}
\institute{Supervised by Dr. Cheng Wan, ETH Zurich \\ Population Ageing and Pension Economics (PAPE)}
\date{December 2, 2025}

\begin{document}

\begin{frame}
  \titlepage
\end{frame}

\begin{frame}{The Demographic Time Bomb is Ticking}
  \begin{itemize}
    \item By 2035, the last of Germany's "Baby Boomer" generation will have retired.
    \item The Old-Age Dependency Ratio (OADR) is projected to soar from 35 to over 50.
    \begin{itemize}
        \item \textbf{Today:} \textasciitilde3 workers support 1 pensioner.
        \item \textbf{By 2050:} \textasciitilde2 workers will have to support 1 pensioner.
    \end{itemize}
    \item \textbf{The Fiscal Squeeze:} Without reform, the contribution rate is forecast to rise from 18.6\% to over 24\% by 2040, while the purchasing power of pensions falls.
  \end{itemize}
  \vspace{1em}
  \centering
  \textbf{\large The "Generational Contract" is under unprecedented stress.}
\end{frame}

\begin{frame}{The Economic Trilemma \& Mackenroth’s Theorem}
  \textbf{The Fundamental Constraint:} The "Pension Trilemma."
  \begin{itemize}
    \item \textbf{Contribution Rate} (\textit{Beitragssatz}): How much workers pay. Politically capped (approx. 20-22\%).
    \item \textbf{Replacement Rate} (\textit{Rentenniveau}): How much pensioners get. Politically protected (floor at 48\%).
    \item \textbf{Retirement Age}: How long you work. Also politically sensitive.
  \end{itemize}
  \textit{You can only fix two; the third must adjust. Germany is trying to fix all three.}
  
  \vspace{1em}
  \textbf{Theoretical Basis:} \textit{Mackenroth’s Theorem (1952)}:
  \begin{quote}
    "All social expenditure must always be paid out of the current national product."
    \\ \small (Goods for retirees are made by current workers, regardless of funding type).
  \end{quote}
\end{frame}

\begin{frame}{Institutional Design: A System Reliant on One Pillar}
  \textbf{Structure:}
  \begin{itemize}
    \item \textbf{Pillar 1 (GRV):} Mandatory Pay-As-You-Go (PAYG).
    \begin{itemize}
        \item Covers \textasciitilde90\% of the workforce.
        \item Provides \textasciitilde75\% of total old-age income.
        \item \textbf{Result:} A "mono-pillar" system, highly exposed to demographic shifts.
    \end{itemize}
    \item \textbf{Pillar 2 (bAV) \& 3 (Private):} Voluntary and have seen limited uptake, unlike in Switzerland or the Netherlands.
  \end{itemize}
  \textbf{Financing:}
  \begin{itemize}
    \item Contributions (Employer/Employee split 50/50).
    \item \textbf{Federal Subsidy (\textit{Bundeszuschuss}):} \textasciitilde30\% of the budget comes from general taxes, not contributions.
  \end{itemize}
\end{frame}

\begin{frame}{Microeconomics: The "Points System" (\textit{Entgeltpunkte})}
  \textbf{The Logic:} Your pension reflects your lifetime relative income, it's not a "final salary" system.
  \begin{itemize}
    \item \textbf{Accumulation Mechanism:}
    \[
      EP_t = \frac{\text{Individual Gross Income}_t}{\text{Average Gross Income}_t}
    \]
    \small{\textbf{Example:} If you earn €50,000 and the average is €50,000, you earn 1.0 point for that year. If you earn €75,000, you get 1.5 points.}
    \item \textbf{Incentive Structure (The Equivalence Principle):}
    \begin{itemize}
        \item Strict linearity ($Contribution \propto Benefit$).
        \item \textbf{Economic Goal:} Makes contributions feel like "deferred wages," not a tax, to minimize labor supply distortions.
    \end{itemize}
  \end{itemize}
\end{frame}

\begin{frame}{The Pension Formula: Turning Points into Euros}
    \textbf{The Equation:}
    \[
        \text{Pension}_{\text{Monthly}} = \sum EP \times ZF \times RAF \times AR
    \]
    \textbf{Variables:}
    \begin{itemize}
        \item $\sum EP$: Your total lifetime earnings points.
        \item $ZF$ (Access Factor): An early retirement penalty. 0.3\% permanent cut for every month you retire before the statutory age.
        \item $RAF$: Pension Type (1.0 for standard old age).
        \item $AR$ (Current Pension Value): The "exchange rate" for one pension point in Euros per month. This value is adjusted periodically.
    \end{itemize}
\end{frame}

\begin{frame}{The Adjustment Mechanism (Sustainability Factor)}
  \textbf{The Goal:} Automatically adjust for demography.
  \begin{itemize}
    \item \textbf{The Dampener: The Sustainability Factor (enacted 2004).}
    \begin{itemize}
      \item If the dependency ratio (pensioners/workers) worsens, pension increases are "dampened" and do not fully follow wage growth.
      \[
        \text{Adjustment}_t \propto \text{WageGrowth}_t \times \left( 1 - \alpha \cdot \frac{\Delta \text{DependencyRatio}_t}{\text{DependencyRatio}_{t-1}} \right)
      \]
      \item \textbf{Function:} It shifts some of the demographic risk from the young (contributors) to the old (retirees).
    \end{itemize}
  \end{itemize}
\end{frame}



\begin{frame}{Labor Supply \& The Politics of Retirement Age}
  \begin{itemize}
    \item \textbf{Policy Response: "Rente mit 67" (2007 Reform).}
    \begin{itemize}
        \item Gradually increases the Statutory Retirement Age (SRA) to 67 by 2031 to keep people working longer.
    \end{itemize}
    \item \textbf{The Policy Anomaly: "Rente mit 63" (2014).}
    \begin{itemize}
        \item Allowed long-term contributors to retire early without penalty.
        \item \textbf{Economic Critique:} A huge "deadweight loss" by subsidizing the exit of highly productive, skilled labor during a growing labor shortage.
    \end{itemize}
    \item \textbf{Result:} The effective retirement age remains below the statutory age.
  \end{itemize}
\end{frame}

\begin{frame}{The "Standard Pensioner" \& Falling Replacement Rates}
  \begin{itemize}
    \item \textbf{Standard Pensioner (\textit{Eckrentner}):} A theoretical person with 45 years of average contributions.
    \begin{itemize}
        \item \textbf{Problem:} This ignores fractured careers, part-time work (especially for women), and unemployment, thus painting a deceptively rosy picture.
    \end{itemize}
    \item \textbf{Replacement Rate Trends (The Worry):}
    \begin{itemize}
        \item The net replacement rate is currently held at \textasciitilde48\%.
        \item \textit{Forecast (Pre-Reform):} The Sustainability Factor would have caused it to drop to <45\% by 2040.
    \end{itemize}
    \item \textbf{Contribution Rate Trends (The Other Worry):}
    \begin{itemize}
        \item Currently \textasciitilde18.6\%.
        \item \textit{Forecast:} Set to break the politically sensitive 20\% barrier by 2028.
    \end{itemize}
  \end{itemize}
\end{frame}

\begin{frame}{Rentenpaket II: Politics Overrules Economics}
  \begin{itemize}
    \item \textbf{The "Double Stop Line" (\textit{Doppelte Haltelinie}):}
    \begin{itemize}
        \item A political promise: Keep the replacement rate $\ge 48\%$ AND the contribution rate $\le 20\%$.
    \end{itemize}
    \item \textbf{The New Deal (Rentenpaket II):}
    \begin{itemize}
        \item Permanently guarantees the 48\% level until 2039.
        \item \textbf{The Cost:} This deactivates the Sustainability Factor, removing the automatic brake on spending.
        \item \textbf{The Consequence:} The fiscal burden is shifted entirely to the federal budget via massive tax subsidies (\textit{Bundeszuschuss}).
    \end{itemize}
  \end{itemize}
\end{frame}

\begin{frame}{Paradigm Shift? The \textit{Generationenkapital}}
    \textbf{Concept:} Create a Sovereign Wealth Fund to help subsidize the pension system.
    \begin{itemize}
        \item \textbf{Mechanism: "Debt-Financed Arbitrage."}
        \begin{itemize}
            \item The state borrows money at low interest rates (cost of government bonds).
            \item It invests this money in a globally diversified portfolio of stocks (aiming for higher equity returns).
            \item The goal is to profit from the spread: $r_{\text{equity}} > r_{\text{bond}}$.
        \end{itemize}
        \item \textbf{Target:} €200bn fund by the mid-2030s.
        \item \textbf{Critique:}
        \begin{itemize}
            \item \textbf{Too Small:} Experts argue it needs >€1 Trillion to have a meaningful impact on contribution rates.
            \item \textbf{Governance Risks:} Can investment decisions remain free from political interference?
        \end{itemize}
    \end{itemize}
\end{frame}

\begin{frame}{The Unseen Debt Burden}
    \begin{itemize}
        \item \textbf{Fiscal Crowding Out:}
        \begin{itemize}
            \item Pension subsidies already consume \textasciitilde25-30\% of the entire German federal budget.
            \item This severely limits fiscal space for infrastructure, defense, digitalization, and education.
        \end{itemize}
        \item \textbf{Implicit Pension Debt:}
        \begin{itemize}
            \item The present value of all pension promises the state has made is enormous: \textbf{over 300\% of Germany's GDP}. This is the "hidden" debt.
        \end{itemize}
        \item \textbf{The "Boomer Voter" Effect:}
        \begin{itemize}
            \item The large voting bloc of the elderly creates strong political resistance to actuarially necessary cuts (Median Voter Theorem).
        \end{itemize}
    \end{itemize}
\end{frame}

\begin{frame}{Conclusion}
  \textbf{Summary:}
  \begin{itemize}
    \item Germany’s PAYG system is efficient in its design but fundamentally vulnerable to its own demographic decline.
    \item Recent reforms (\textit{Rentenpaket II}) have prioritized short-term benefit security for current pensioners over long-term fiscal sustainability, effectively passing the bill to future generations.
  \end{itemize}
  \textbf{The Outlook:}
  \begin{itemize}
    \item The \textit{Generationenkapital} is a historic step towards capital funding, but it is too small to solve the structural problem.
    \item The "Contract between Generations" is being rewritten, with the young bearing the demographic and fiscal risk.
  \end{itemize}
  \vspace{1em}
  \centering
  \textbf{\large The final question remains: Can productivity growth outpace the demographic drag?}
\end{frame}

\begin{frame}{References}
  \begin{itemize}
    \item German Council of Economic Experts (Sachverständigenrat).
    \item Deutsche Rentenversicherung (DRV) Reports 2023/2024.
    \item OECD Pensions at a Glance 2023.
    \item Mackenroth, G. (1952). \textit{Die Reform der Sozialpolitik}.
    \item \textit{Note: Comparisons to Swiss and Japanese systems drawn from course materials}.
  \end{itemize}
\end{frame}

\end{document}
